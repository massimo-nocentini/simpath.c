\input cwebmac
\datethis




\N{1}{1}Introduction. This program inputs a directed graph.
It outputs a not-necessarily-reduced binary decision diagram
for the family of all simple oriented cycles in that graph.

The format of the output is described in another program,
{\mc SIMPATH-REDUCE}. Let me just say here that it is intended
only for computational convenience, not for human readability.

I've tried to make this program simple, whenever I had to
choose between simplicity and efficiency. But I haven't gone
out of my way to be inefficient.

(Notes, 30 November 2015: My original version of this program,
written in August 2008, was hacked from {\mc SIMPATH}. I~don't
think I used it much at that time, if at all, because
I made a change in February 2010 to make it compile without
errors. Today I'm making two fundamental changes:
(i) Each ``frontier'' in {\mc SIMPATH} was required
to be an interval of vertices, according to the vertex numbering.
Now the elements of each frontier are listed explicitly; so
I needn't waste space by including elements that don't really
participate in frontier activities. (ii)~I do {\it not\/}
renumber the vertices.
The main advantage of these two changes is
that I can put a dummy vertex at the end, with arcs to and from
every other vertex; then we get all the simple {\it paths\/} instead
of all the simple {\it cycles}, while the frontiers stay the same size
except for the dummy element. And we can modify this program to get all
the oriented {\it Hamiltonian\/} paths as well.)

\Y\B\4\D$\\{maxn}$ \5
\T{90}\C{ maximum number of vertices; at most 126 }\par
\B\4\D$\\{maxm}$ \5
\T{2000}\C{ maximum number of arcs }\par
\B\4\D$\\{logmemsize}$ \5
\T{27}\par
\B\4\D$\\{memsize}$ \5
$(\T{1}\LL\\{logmemsize}{}$)\par
\B\4\D$\\{loghtsize}$ \5
\T{24}\par
\B\4\D$\\{htsize}$ \5
$(\T{1}\LL\\{loghtsize}{}$)\par
\Y\B\8\#\&{include} \.{<stdio.h>}\6
\8\#\&{include} \.{<stdlib.h>}\6
\8\#\&{include} \.{<string.h>}\6
\8\#\&{include} \.{"gb\_graph.h"}\6
\8\#\&{include} \.{"gb\_save.h"}\6
\&{char} \\{mem}[\\{memsize}];\C{ the big workspace }\6
\&{unsigned} \&{long} \&{long} \\{tail}${},\39\\{boundary},\39\\{head}{}$;\C{
queue pointers }\6
\&{unsigned} \&{int} \\{htable}[\\{htsize}];\C{ hash table }\6
\&{unsigned} \&{int} \\{htid};\C{ ``time stamp'' for hash entries }\6
\&{int} \\{htcount};\C{ number of entries in the hash table }\6
\&{int} \\{wrap}${}\K\T{1}{}$;\C{ wraparound counter for hash table clearing }\6
\&{Vertex} ${}{*}\\{vert}[\\{maxn}+\T{1}];{}$\6
\&{int} ${}\|f[\\{maxn}+\T{2}],\39\\{ff}[\\{maxn}+\T{2}]{}$;\C{ elements of the
current and the next frontier }\6
\&{int} \|s${},\39\\{ss}{}$;\C{ the sizes of \PB{\|f} and \PB{\\{ff}} }\6
\&{int} ${}\\{curfront}[\\{maxn}+\T{1}],\39\\{nextfront}[\\{maxn}+\T{1}]{}$;\C{
inverse frontier map }\6
\&{int} \\{arcto}[\\{maxm}];\C{ destination number of each arc }\6
\&{int} ${}\\{firstarc}[\\{maxn}+\T{2}]{}$;\C{ where arcs from a vertex start
in \PB{\\{arcto}} }\6
\&{char} ${}\\{mate}[\\{maxn}+\T{3}]{}$;\C{ encoded state }\6
\&{int} \\{serial}${},\39\\{newserial}{}$;\C{ state numbers }\7
\X13:Subroutines\X\7
\1\1\\{main}(\&{int} \\{argc}${},\39{}$\&{char} ${}{*}\\{argv}[\,])\2\2{}$\6
${}\{{}$\1\6
\&{register} \&{int} \|i${},\39\|j,\39\\{jj},\39\\{jm},\39\|k,\39\\{km},\39\|l,%
\39\\{ll},\39\|m,\39\|n,\39\|p,\39\|t,\39\\{hash},\39\\{sign};{}$\6
\&{register} \&{Graph} ${}{*}\|g;{}$\6
\&{register} \&{Arc} ${}{*}\|a,\39{*}\|b;{}$\6
\&{register} \&{Vertex} ${}{*}\|u,\39{*}\|v;{}$\7
\X2:Input the graph\X;\6
\X3:Reformat the arcs\X;\6
\X5:Do the algorithm\X;\6
\4${}\}{}$\2\par
\fi

\M{2}\B\X2:Input the graph\X${}\E{}$\6
\&{if} ${}(\\{argc}\I\T{2}){}$\5
${}\{{}$\1\6
${}\\{fprintf}(\\{stderr},\39\.{"Usage:\ \%s\ foo.gb\\n"},\39\\{argv}[%
\T{0}]);{}$\6
${}\\{exit}({-}\T{1});{}$\6
\4${}\}{}$\2\6
${}\|g\K\\{restore\_graph}(\\{argv}[\T{1}]);{}$\6
\&{if} ${}(\R\|g){}$\5
${}\{{}$\1\6
${}\\{fprintf}(\\{stderr},\39\.{"I\ can't\ input\ the\ g}\)\.{raph\ \%s\ (panic%
\ code\ }\)\.{\%ld)!\\n"},\39\\{argv}[\T{1}],\39\\{panic\_code});{}$\6
${}\\{exit}({-}\T{2});{}$\6
\4${}\}{}$\2\6
${}\|n\K\|g\MG\|n;{}$\6
\&{if} ${}(\|n>\\{maxn}){}$\5
${}\{{}$\1\6
${}\\{fprintf}(\\{stderr},\39\.{"Sorry,\ that\ graph\ h}\)\.{as\ \%d\ vertices;%
\ "},\39\|n);{}$\6
${}\\{fprintf}(\\{stderr},\39\.{"I\ can't\ handle\ more}\)\.{\ than\ \%d!\\n"},%
\39\\{maxn});{}$\6
${}\\{exit}({-}\T{3});{}$\6
\4${}\}{}$\2\6
\&{if} ${}(\|g\MG\|m>\\{maxm}){}$\5
${}\{{}$\1\6
${}\\{fprintf}(\\{stderr},\39\.{"Sorry,\ that\ graph\ h}\)\.{as\ \%ld\ arcs;\
"},\39(\|g\MG\|m+\T{1})/\T{2});{}$\6
${}\\{fprintf}(\\{stderr},\39\.{"I\ can't\ handle\ more}\)\.{\ than\ \%d!\\n"},%
\39\\{maxm});{}$\6
${}\\{exit}({-}\T{3});{}$\6
\4${}\}{}$\2\par
\U1.\fi

\M{3}The arcs will be either $j\to k$ or $j\gets k$ between vertex number~$j$
and vertex number~$k$, when $j<k$ and those vertices are adjacent in
the graph. We process them in order of increasing~$j$; but for fixed~$j$,
the values of~$k$ aren't necessarily increasing.

The $k$ values appear in the \PB{\\{arcto}} array, with $-k$ used for the arcs
that emanate from~$k$. The arcs for fixed~$j$
occur in positions \PB{\\{firstarc}[\|j]} through \PB{$\\{firstarc}[\|j+\T{1}]-%
\T{1}$} of that array.

After this step, we forget the GraphBase data structures and just work
with our homegrown integer-only representation.

\Y\B\4\X3:Reformat the arcs\X${}\E{}$\6
\X4:Make the inverse-arc lists\X;\6
\&{for} ${}(\|m\K\T{0},\39\|k\K\T{1};{}$ ${}\|k\Z\|n;{}$ ${}\|k\PP){}$\5
${}\{{}$\1\6
${}\\{firstarc}[\|k]\K\|m;{}$\6
${}\|v\K\\{vert}[\|k];{}$\6
${}\\{printf}(\.{"\%d(\%s)\\n"},\39\|k,\39\|v\MG\\{name});{}$\6
\&{for} ${}(\|a\K\|v\MG\\{arcs};{}$ \|a; ${}\|a\K\|a\MG\\{next}){}$\5
${}\{{}$\1\6
${}\|u\K\|a\MG\\{tip};{}$\6
\&{if} ${}(\|u>\|v){}$\5
${}\{{}$\1\6
${}\\{arcto}[\|m\PP]\K\|u-\|g\MG\\{vertices}+\T{1};{}$\6
\&{if} ${}(\|a\MG\\{len}\E\T{1}){}$\1\5
${}\\{printf}(\.{"\ ->\ \%ld(\%s)\ \#\%d\\n"},\39\|u-\|g\MG\\{vertices}+\T{1},%
\39\|u\MG\\{name},\39\|m);{}$\2\6
\&{else}\1\5
${}\\{printf}(\.{"\ ->\ \%ld(\%s,\%ld)\ \#\%d}\)\.{\\n"},\39\|u-\|g\MG%
\\{vertices}+\T{1},\39\|u\MG\\{name},\39\|a\MG\\{len},\39\|m);{}$\2\6
\4${}\}{}$\2\6
\4${}\}{}$\2\6
\&{for} ${}(\|a\K\|v\MG\\{invarcs};{}$ \|a; ${}\|a\K\|a\MG\\{next}){}$\5
${}\{{}$\1\6
${}\|u\K\|a\MG\\{tip};{}$\6
\&{if} ${}(\|u>\|v){}$\5
${}\{{}$\1\6
${}\\{arcto}[\|m\PP]\K{-}(\|u-\|g\MG\\{vertices}+\T{1});{}$\6
\&{if} ${}(\|a\MG\\{len}\E\T{1}){}$\1\5
${}\\{printf}(\.{"\ <-\ \%ld(\%s)\ \#\%d\\n"},\39\|u-\|g\MG\\{vertices}+\T{1},%
\39\|u\MG\\{name},\39\|m);{}$\2\6
\&{else}\1\5
${}\\{printf}(\.{"\ <-\ \%ld(\%s,\%ld)\ \#\%d}\)\.{\\n"},\39\|u-\|g\MG%
\\{vertices}+\T{1},\39\|u\MG\\{name},\39\|a\MG\\{len},\39\|m);{}$\2\6
\4${}\}{}$\2\6
\4${}\}{}$\2\6
\4${}\}{}$\2\6
${}\\{firstarc}[\|k]\K\|m{}$;\par
\U1.\fi

\M{4}To aid in the desired sorting, we first create an inverse-arc
list for each vertex~\PB{\|v}, namely a list of vertices that point to~\PB{%
\|v}.

\Y\B\4\D$\\{invarcs}$ \5
$\|y.{}$\|A\par
\Y\B\4\X4:Make the inverse-arc lists\X${}\E{}$\6
\&{for} ${}(\|v\K\|g\MG\\{vertices};{}$ ${}\|v<\|g\MG\\{vertices}+\|n;{}$ ${}%
\|v\PP){}$\1\5
${}\|v\MG\\{invarcs}\K\NULL;{}$\2\6
\&{for} ${}(\|v\K\|g\MG\\{vertices};{}$ ${}\|v<\|g\MG\\{vertices}+\|n;{}$ ${}%
\|v\PP){}$\5
${}\{{}$\1\6
${}\\{vert}[\|v-\|g\MG\\{vertices}+\T{1}]\K\|v;{}$\6
\&{for} ${}(\|a\K\|v\MG\\{arcs};{}$ \|a; ${}\|a\K\|a\MG\\{next}){}$\5
${}\{{}$\1\6
\&{register} \&{Arc} ${}{*}\|b\K\\{gb\_virgin\_arc}(\,);{}$\7
${}\|u\K\|a\MG\\{tip};{}$\6
${}\|b\MG\\{tip}\K\|v;{}$\6
${}\|b\MG\\{len}\K\|a\MG\\{len};{}$\6
${}\|b\MG\\{next}\K\|u\MG\\{invarcs};{}$\6
${}\|u\MG\\{invarcs}\K\|b;{}$\6
\4${}\}{}$\2\6
\4${}\}{}$\2\par
\U3.\fi

\N{1}{5}The algorithm.
Now comes the fun part. We systematically construct a binary decision
diagram for all simple paths by working top-down, considering the
arcs in \PB{\\{arcto}}, one by one.

When we're dealing with arc \PB{\|i}, we've already constructed a table of
all possible states that might arise when each of the previous arcs has
been chosen-or-not, except for states that obviously cannot be
part of a simple path.

Arc \PB{\|i} runs from vertex \PB{\|j} to vertex \PB{$\|k\K\\{arcto}[\|i]$},
or from \PB{$\|k\K{-}\\{arcto}[\|i]$} to~\PB{\|j}.

Let $F_i=\{v_1,\ldots,v_s\}$ be the {\it frontier\/} at arc~\PB{\|i},
namely the set of vertex numbers \PB{$\G$ \|j} that appear in arcs~\PB{$<$ %
\|i}.

The state before we decide whether or not to include arc~\PB{\|i} is
represented by a table of values \PB{\\{mate}[\|t]}, for $t\in F_i\cup\{j,k\}$,
with the following significance:
If \PB{$\\{mate}[\|t]\K\|t$}, the previous arcs haven't touched vertex \PB{%
\|t}.
If \PB{$\\{mate}[\|t]\K\|u$} and \PB{$\|u\I\|t$}, the previous arcs have made a
simple directed
path from \PB{\|t} to \PB{\|u}.
If \PB{$\\{mate}[\|t]\K{-}\|u$}, the previous arcs have made a simple directed
path from \PB{\|u} to \PB{\|t}.
If \PB{$\\{mate}[\|t]\K\T{0}$}, the previous arcs have ``saturated'' vertex~%
\PB{\|t}; we can't
touch it again.

The \PB{\\{mate}} information is all that we need to know about the behavior of
previous arcs. And it's easily updated when we add the \PB{\|i}th arc (or not).
So each ``state'' is equivalent to a \PB{\\{mate}} table, consisting of
\PB{\|s} numbers, where $s$ is the size of~$F_i$.

The states are stored in a queue, indexed by 64-bit numbers
\PB{\\{tail}}, \PB{\\{boundary}}, and \PB{\\{head}}, where \PB{$\\{tail}\Z%
\\{boundary}\Z\\{head}$}.
Between \PB{\\{tail}} and \PB{\\{boundary}} are the pre-arc-\PB{\|i} states
that haven't yet
been processed; between \PB{\\{boundary}} and \PB{\\{head}} are the post-arc-%
\PB{\|i} states
that will be considered later. The states before \PB{\\{boundary}}
are sequences of \PB{\|s} bytes each, and the states after \PB{\\{boundary}}
are sequences of \PB{\\{ss}} bytes each, where \PB{\\{ss}} is the size
of~$F_{i+1}$.

(Exception: If \PB{$\|s\K\T{0}$}, we use one byte to represent the state,
although
we ignore it when reading from the queue later. In this way
we know how many states are present.)

Bytes of the queue are stored in \PB{\\{mem}}, which wraps around modulo \PB{%
\\{memsize}}.
We ensure that \PB{$\\{head}-\\{tail}$} never exceeds \PB{\\{memsize}}.

\Y\B\4\X5:Do the algorithm\X${}\E{}$\6
\&{for} ${}(\|t\K\T{1};{}$ ${}\|t\Z\|n;{}$ ${}\|t\PP){}$\1\5
${}\\{mate}[\|t]\K\|t;{}$\2\6
\X6:Initialize the queue\X;\6
\&{for} ${}(\|i\K\T{0};{}$ ${}\|i<\|m;{}$ ${}\|i\PP){}$\5
${}\{{}$\1\6
${}\\{printf}(\.{"\#\%d:\\n"},\39\|i+\T{1}){}$;\C{ announce that we're
beginning a new arc }\6
${}\\{fprintf}(\\{stderr},\39\.{"Beginning\ arc\ \%d\ (s}\)\.{erial=%
\%d,head-tail=\%}\)\.{lld)\\n"},\39\|i+\T{1},\39\\{serial},\39\\{head}-%
\\{tail});{}$\6
\\{fflush}(\\{stderr});\6
\X7:Process arc \PB{\|i}\X;\6
\4${}\}{}$\2\6
${}\\{printf}(\.{"\%x:0,0\\n"},\39\\{serial}){}$;\par
\U1.\fi

\M{6}Each state for a particular arc gets a distinguishing number, where
its ZDD instructions begin.
Two states are special: 0 means the losing state, when a simple path
is impossible; 1 means the winning state, when a simple path has been
completed. The other states are 2 or more.

Initially \PB{\|i} will be zero, and the queue is empty. We'll want
\PB{\\{jj}} to be the the \PB{\|j} vertex of arc \PB{$\|i+\T{1}$}, and \PB{%
\\{ss}} to be the
size of~$F_{i+1}$. Also \PB{\\{serial}} is the identifying number for
arc~\PB{$\|i+\T{1}$}.

\Y\B\4\X6:Initialize the queue\X${}\E{}$\6
$\\{jj}\K\T{1},\39\\{ss}\K\T{0};{}$\6
\&{while} ${}(\\{firstarc}[\\{jj}+\T{1}]\E\T{0}){}$\1\5
${}\\{jj}\PP{}$;\C{ unnecessary unless vertex 1 is isolated }\2\6
${}\\{tail}\K\\{head}\K\T{0};{}$\6
${}\\{serial}\K\T{2}{}$;\par
\U5.\fi

\M{7}The output format on \PB{\\{stdout}} simply shows the identifying numbers
of a state
and its two successors, in hexadecimal.

\Y\B\4\D$\\{trunc}(\\{addr})$ \5
$((\\{addr})\AND(\\{memsize}-\T{1}){}$)\par
\Y\B\4\X7:Process arc \PB{\|i}\X${}\E{}$\6
\&{if} ${}(\\{ss}\E\T{0}){}$\1\5
${}\\{head}\PP{}$;\C{ put a dummy byte into the queue }\2\6
${}\\{boundary}\K\\{head},\39\\{htcount}\K\T{0},\39\\{htid}\K(\|i+\\{wrap})\LL%
\\{logmemsize};{}$\6
\&{if} ${}(\\{htid}\E\T{0}){}$\5
${}\{{}$\1\6
\&{for} ${}(\\{hash}\K\T{0};{}$ ${}\\{hash}<\\{htsize};{}$ ${}\\{hash}\PP){}$\1%
\5
${}\\{htable}[\\{hash}]\K\T{0};{}$\2\6
${}\\{wrap}\PP,\39\\{htid}\K\T{1}\LL\\{logmemsize};{}$\6
\4${}\}{}$\2\6
${}\\{newserial}\K\\{serial}+(\\{head}-\\{tail})/(\\{ss}\?\\{ss}:\T{1});{}$\6
${}\|j\K\\{jj},\39\\{sign}\K\\{arcto}[\|i],\39\|k\K(\\{sign}>\T{0}\?%
\\{sign}:{-}\\{sign}),\39\|s\K\\{ss};{}$\6
\&{for} ${}(\|p\K\T{0};{}$ ${}\|p<\|s;{}$ ${}\|p\PP){}$\1\5
${}\|f[\|p]\K\\{ff}[\|p];{}$\2\6
\X8:Compute \PB{\\{jj}} and $F_{i+1}$\X;\6
\&{while} ${}(\\{tail}<\\{boundary}){}$\5
${}\{{}$\1\6
${}\\{printf}(\.{"\%x:"},\39\\{serial});{}$\6
${}\\{serial}\PP;{}$\6
\X9:Unpack a state, and move \PB{\\{tail}} up\X;\6
\X11:Print the successor if arc \PB{\|i} is not chosen\X;\6
\\{printf}(\.{","});\6
\X10:Print the successor if arc \PB{\|i} is chosen\X;\6
\\{printf}(\.{"\\n"});\6
\4${}\}{}$\2\par
\U5.\fi

\M{8}Here we set \PB{\\{nextfront}[\|t]} to \PB{$\|i+\T{1}$} whenever $t\in
F_{i+1}$.
And we also set \PB{\\{curfront}[\|t]} to \PB{$\|i+\T{1}$} wheneer $t\in F_i$;
I~use \PB{$\|i+\T{1}$}, not~\PB{\|i}, because the \PB{\\{curfront}} array is
initially zero.

\Y\B\4\X8:Compute \PB{\\{jj}} and $F_{i+1}$\X${}\E{}$\6
\&{while} ${}(\\{jj}\Z\|n\W\\{firstarc}[\\{jj}+\T{1}]\E\|i+\T{1}){}$\1\5
${}\\{jj}\PP;{}$\2\6
\&{for} ${}(\|p\K\\{ss}\K\T{0};{}$ ${}\|p<\|s;{}$ ${}\|p\PP){}$\5
${}\{{}$\1\6
${}\|t\K\|f[\|p];{}$\6
${}\\{curfront}[\|t]\K\|i+\T{1};{}$\6
\&{if} ${}(\|t\G\\{jj}){}$\5
${}\{{}$\1\6
${}\\{nextfront}[\|t]\K\|i+\T{1};{}$\6
${}\\{ff}[\\{ss}\PP]\K\|t;{}$\6
\4${}\}{}$\2\6
\4${}\}{}$\2\6
\&{if} ${}(\|j\E\\{jj}\W\\{nextfront}[\|j]\I\|i+\T{1}){}$\1\5
${}\\{nextfront}[\|j]\K\|i+\T{1},\39\\{ff}[\\{ss}\PP]\K\|j;{}$\2\6
\&{if} ${}(\|k\G\\{jj}\W\\{nextfront}[\|k]\I\|i+\T{1}){}$\1\5
${}\\{nextfront}[\|k]\K\|i+\T{1},\39\\{ff}[\\{ss}\PP]\K\|k{}$;\2\par
\U7.\fi

\M{9}This step sets \PB{\\{mate}[\|t]} for all $t\in F_i\cup\{j,k\}$, based on
a
queued state, while taking \PB{\|s} bytes out of the queue.

\Y\B\4\X9:Unpack a state, and move \PB{\\{tail}} up\X${}\E{}$\6
\&{if} ${}(\|s\E\T{0}){}$\1\5
${}\\{tail}\PP;{}$\2\6
\&{else}\5
${}\{{}$\1\6
\&{for} ${}(\|p\K\T{0};{}$ ${}\|p<\|s;{}$ ${}\|p\PP,\39\\{tail}\PP){}$\5
${}\{{}$\1\6
${}\|t\K\|f[\|p];{}$\6
${}\\{mate}[\|t]\K\\{mem}[\\{trunc}(\\{tail})];{}$\6
\4${}\}{}$\2\6
\4${}\}{}$\2\6
\&{if} ${}(\\{curfront}[\|j]\I\|i+\T{1}){}$\1\5
${}\\{mate}[\|j]\K\|j;{}$\2\6
\&{if} ${}(\\{curfront}[\|k]\I\|i+\T{1}){}$\1\5
${}\\{mate}[\|k]\K\|k{}$;\2\par
\U7.\fi

\M{10}Here's where we update the mates. The order of doing this is carefully
chosen so that it works fine when \PB{$\\{mate}[\|j]\K\|j$} and/or \PB{$%
\\{mate}[\|k]\K\|k$}.

\Y\B\4\X10:Print the successor if arc \PB{\|i} is chosen\X${}\E{}$\6
\&{if} ${}(\\{sign}>\T{0}){}$\5
${}\{{}$\1\6
${}\\{jm}\K\\{mate}[\|j],\39\\{km}\K\\{mate}[\|k];{}$\6
\&{if} ${}(\\{jm}\E\|j){}$\1\5
${}\\{jm}\K{-}\|j;{}$\2\6
\&{if} ${}(\\{jm}\G\T{0}\V\\{km}\Z\T{0}){}$\1\5
\\{printf}(\.{"0"});\C{ we mustn't touch a saturated vertex }\2\6
\&{else} \&{if} ${}(\\{jm}\E{-}\|k){}$\1\5
\X12:Print 1 or 0, depending on whether this arc wins or loses\X\2\6
\&{else}\5
${}\{{}$\1\6
${}\\{mate}[\|j]\K\T{0},\39\\{mate}[\|k]\K\T{0};{}$\6
${}\\{mate}[{-}\\{jm}]\K\\{km},\39\\{mate}[\\{km}]\K\\{jm};{}$\6
${}\\{printstate}(\|j,\39\\{jj},\39\|i,\39\|k);{}$\6
\4${}\}{}$\2\6
\4${}\}{}$\5
\2\&{else}\5
${}\{{}$\1\6
${}\\{jm}\K\\{mate}[\|j],\39\\{km}\K\\{mate}[\|k];{}$\6
\&{if} ${}(\\{km}\E\|k){}$\1\5
${}\\{km}\K{-}\|k;{}$\2\6
\&{if} ${}(\\{jm}\Z\T{0}\V\\{km}\G\T{0}){}$\1\5
\\{printf}(\.{"0"});\C{ we mustn't touch a saturated vertex }\2\6
\&{else} \&{if} ${}(\\{km}\E{-}\|j){}$\1\5
\X12:Print 1 or 0, depending on whether this arc wins or loses\X\2\6
\&{else}\5
${}\{{}$\1\6
${}\\{mate}[\|j]\K\T{0},\39\\{mate}[\|k]\K\T{0};{}$\6
${}\\{mate}[\\{jm}]\K\\{km},\39\\{mate}[{-}\\{km}]\K\\{jm};{}$\6
${}\\{printstate}(\|j,\39\\{jj},\39\|i,\39\|k);{}$\6
\4${}\}{}$\2\6
\4${}\}{}$\2\par
\U7.\fi

\M{11}\B\X11:Print the successor if arc \PB{\|i} is not chosen\X${}\E{}$\6
$\\{printstate}(\|j,\39\\{jj},\39\|i,\39\|k){}$;\par
\U7.\fi

\M{12}See the note below regarding a change that will restrict consideration
to Hamiltonian paths. A similar change is needed here.

\Y\B\4\X12:Print 1 or 0, depending on whether this arc wins or loses\X${}\E{}$\6
${}\{{}$\1\6
\&{for} ${}(\|p\K\T{0};{}$ ${}\|p<\|s;{}$ ${}\|p\PP){}$\5
${}\{{}$\1\6
${}\|t\K\|f[\|p];{}$\6
\&{if} ${}(\|t\I\|j\W\|t\I\|k\W\\{mate}[\|t]\W\\{mate}[\|t]\I\|t){}$\1\5
\&{break};\2\6
\4${}\}{}$\2\6
\&{if} ${}(\|p\E\|s){}$\1\5
\\{printf}(\.{"1"});\C{ we win: this cycle is all by itself }\2\6
\&{else}\1\5
\\{printf}(\.{"0"});\C{ we lose: there's junk outside this cycle }\2\6
\4${}\}{}$\2\par
\U10.\fi

\M{13}The \PB{\\{printstate}} subroutine does the rest of the work. It makes
sure
that no incomplete paths linger in positions that are about to disappear
from the current frontier; and it puts the \PB{\\{mate}} entries of the next
frontier
into the queue, checking to see if that state was already there.

If `\PB{$\\{mate}[\|t]\I\|t$}' is removed from the condition below, we get
Hamiltonian cycles only (I mean, simple cycles that include every vertex).

\Y\B\4\X13:Subroutines\X${}\E{}$\6
\1\1\&{void} \\{printstate}(\&{int} \|j${},\39{}$\&{int} \\{jj}${},\39{}$%
\&{int} \|i${},\39{}$\&{int} \|k)\2\2\6
${}\{{}$\1\6
\&{register} \&{int} \|h${},\39\\{hh},\39\|p,\39\|t,\39\\{tt},\39\\{hash};{}$\7
\&{for} ${}(\|p\K\T{0};{}$ ${}\|p<\|s;{}$ ${}\|p\PP){}$\5
${}\{{}$\1\6
${}\|t\K\|f[\|p];{}$\6
\&{if} ${}(\\{nextfront}[\|t]\I\|i+\T{1}\W\\{mate}[\|t]\W\\{mate}[\|t]\I\|t){}$%
\1\5
\&{break};\2\6
\4${}\}{}$\2\6
\&{if} ${}(\|p<\|s){}$\1\5
\\{printf}(\.{"0"});\C{ incomplete junk mustn't be left hanging }\2\6
\&{else} \&{if} ${}(\\{nextfront}[\|j]\I\|i+\T{1}\W\\{mate}[\|j]\W\\{mate}[\|j]%
\I\|j){}$\1\5
\\{printf}(\.{"0"});\2\6
\&{else} \&{if} ${}(\\{nextfront}[\|k]\I\|i+\T{1}\W\\{mate}[\|k]\W\\{mate}[\|k]%
\I\|k){}$\1\5
\\{printf}(\.{"0"});\2\6
\&{else} \&{if} ${}(\\{ss}\E\T{0}){}$\1\5
${}\\{printf}(\.{"\%x"},\39\\{newserial});{}$\2\6
\&{else}\5
${}\{{}$\1\6
\&{if} ${}(\\{head}+\\{ss}-\\{tail}>\\{memsize}){}$\5
${}\{{}$\1\6
${}\\{fprintf}(\\{stderr},\39\.{"Oops,\ I'm\ out\ of\ me}\)\.{mory:\ memsize=%
\%d,\ se}\)\.{rial=\%d!\\n"},\39\\{memsize},\39\\{serial});{}$\6
${}\\{exit}({-}\T{69});{}$\6
\4${}\}{}$\2\6
\X14:Move the current state into position after \PB{\\{head}}, and compute \PB{%
\\{hash}}\X;\6
\X15:Find the first match, \PB{\\{hh}}, for the current state after \PB{%
\\{boundary}}\X;\6
${}\|h\K\\{trunc}(\\{hh}-\\{boundary})/\\{ss};{}$\6
${}\\{printf}(\.{"\%x"},\39\\{newserial}+\|h);{}$\6
\4${}\}{}$\2\6
\4${}\}{}$\2\par
\U1.\fi

\M{14}\B\X14:Move the current state into position after \PB{\\{head}}, and
compute \PB{\\{hash}}\X${}\E{}$\6
\&{for} ${}(\|p\K\T{0},\39\|h\K\\{trunc}(\\{head}),\39\\{hash}\K\T{0};{}$ ${}%
\|p<\\{ss};{}$ ${}\|p\PP,\39\|h\K\\{trunc}(\|h+\T{1})){}$\5
${}\{{}$\1\6
${}\|t\K\\{ff}[\|p];{}$\6
${}\\{mem}[\|h]\K\\{mate}[\|t];{}$\6
${}\\{hash}\K\\{hash}*\T{31415926525}+\\{mate}[\|t];{}$\6
\4${}\}{}$\2\par
\U13.\fi

\M{15}The hash table is automatically cleared whenever \PB{\\{htid}} is
increased,
because we store \PB{\\{htid}} with each relevant table entry.

\Y\B\4\X15:Find the first match, \PB{\\{hh}}, for the current state after \PB{%
\\{boundary}}\X${}\E{}$\6
\&{for} ${}(\\{hash}\K\\{hash}\AND(\\{htsize}-\T{1});{}$  ; ${}\\{hash}\K(%
\\{hash}+\T{1})\AND(\\{htsize}-\T{1})){}$\5
${}\{{}$\1\6
${}\\{hh}\K\\{htable}[\\{hash}];{}$\6
\&{if} ${}((\\{hh}\XOR\\{htid})\G\\{memsize}){}$\1\5
\X16:Insert new entry and \PB{\&{goto} \\{found}}\X;\2\6
${}\\{hh}\K\\{trunc}(\\{hh});{}$\6
\&{for} ${}(\|t\K\\{hh},\39\|h\K\\{trunc}(\\{head}),\39\\{tt}\K\\{trunc}(\|t+%
\\{ss}-\T{1});{}$  ; ${}\|t\K\\{trunc}(\|t+\T{1}),\39\|h\K\\{trunc}(\|h+%
\T{1})){}$\5
${}\{{}$\1\6
\&{if} ${}(\\{mem}[\|t]\I\\{mem}[\|h]){}$\1\5
\&{break};\2\6
\&{if} ${}(\|t\E\\{tt}){}$\1\5
\&{goto} \\{found};\2\6
\4${}\}{}$\2\6
\4${}\}{}$\2\6
\\{found}:\par
\U13.\fi

\M{16}\B\X16:Insert new entry and \PB{\&{goto} \\{found}}\X${}\E{}$\6
${}\{{}$\1\6
\&{if} ${}(\PP\\{htcount}>(\\{htsize}\GG\T{1})){}$\5
${}\{{}$\1\6
${}\\{fprintf}(\\{stderr},\39\.{"Sorry,\ the\ hash\ tab}\)\.{le\ is\ full\
(htsize=\%}\)\.{d,\ serial=\%d)!\\n"},\39\\{htsize},\39\\{serial});{}$\6
${}\\{exit}({-}\T{96});{}$\6
\4${}\}{}$\2\6
${}\\{hh}\K\\{trunc}(\\{head});{}$\6
${}\\{htable}[\\{hash}]\K\\{htid}+\\{hh};{}$\6
${}\\{head}\MRL{+{\K}}\\{ss};{}$\6
\&{goto} \\{found};\6
\4${}\}{}$\2\par
\U15.\fi

\N{1}{17}Index.
\fi

\inx
\fin
\con
