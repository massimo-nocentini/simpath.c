\input cwebmac
\datethis




\N{1}{1\*}Introduction. This program inputs an undirected graph and the names
of two
vertices in that graph (the ``source'' and ``target'' vertices).
It outputs a not-necessarily-reduced binary decision diagram
for the family of all simple paths from the source to the target.

The format of the output is described in another program,
{\mc SIMPATH-REDUCE}. Let me just say here that it is intended
only for computational convenience, not for human readability.

I've tried to make this program simple, whenever I had to
choose between simplicity and efficiency. But I haven't gone
out of my way to be inefficient.

\Y\B\4\D$\\{maxn}$ \5
\T{254}\C{ maximum number of vertices; at most 254 }\par
\B\4\D$\\{dummy}$ \5
$\\{maxn}+{}$\T{1}\C{ code for a dummy vertex adjacent to all }\par
\B\4\D$\\{maxm}$ \5
\T{2000}\C{ maximum number of edges }\par
\B\4\D$\\{logmemsize}$ \5
\T{27}\par
\B\4\D$\\{memsize}$ \5
$(\T{1}\LL\\{logmemsize}{}$)\par
\B\4\D$\\{loghtsize}$ \5
\T{25}\par
\B\4\D$\\{htsize}$ \5
$(\T{1}\LL\\{loghtsize}{}$)\par
\Y\B\8\#\&{include} \.{<stdio.h>}\6
\8\#\&{include} \.{<stdlib.h>}\6
\8\#\&{include} \.{<string.h>}\6
\8\#\&{include} \.{"gb\_graph.h"}\6
\8\#\&{include} \.{"gb\_save.h"}\6
\&{unsigned} \&{char} \\{mem}[\\{memsize}];\C{ the big workspace }\6
\&{unsigned} \&{long} \&{long} \\{tail}${},\39\\{boundary},\39\\{head}{}$;\C{
queue pointers }\6
\&{unsigned} \&{int} \\{htable}[\\{htsize}];\C{ hash table }\6
\&{unsigned} \&{int} \\{htid};\C{ ``time stamp'' for hash entries }\6
\&{int} \\{htcount};\C{ number of entries in the hash table }\6
\&{int} \\{wrap}${}\K\T{1}{}$;\C{ wraparound counter for hash table clearing }\6
\&{Vertex} ${}{*}\\{vert}[\\{maxn}+\T{1}];{}$\6
\&{int} \\{arcto}[\\{maxm}];\C{ destination number of each arc }\6
\&{int} ${}\\{firstarc}[\\{maxn}+\T{2}]{}$;\C{ where arcs from a vertex start
in \PB{\\{arcto}} }\6
\&{unsigned} \&{char} ${}\\{mate}[\\{maxn}+\T{3}]{}$;\C{ encoded state }\6
\&{int} \\{serial}${},\39\\{newserial}{}$;\C{ state numbers }\7
\X13\*:Subroutines\X\7
\1\1\\{main}(\&{int} \\{argc}${},\39{}$\&{char} ${}{*}\\{argv}[\,])\2\2{}$\6
${}\{{}$\1\6
\&{register} \&{int} \|i${},\39\|j,\39\\{jj},\39\\{jm},\39\|k,\39\\{km},\39\|l,%
\39\\{ll},\39\|m,\39\|n,\39\|t,\39\\{hash};{}$\6
\&{register} \&{Graph} ${}{*}\|g;{}$\6
\&{register} \&{Arc} ${}{*}\|a,\39{*}\|b;{}$\6
\&{register} \&{Vertex} ${}{*}\|u,\39{*}\|v;{}$\6
\&{Vertex} ${}{*}\\{source}\K\NULL,\39{*}\\{target}\K\NULL;{}$\7
\X2:Input the graph\X;\6
\X3:Renumber the vertices\X;\6
\X4\*:Reformat the edges\X;\6
\X5:Do the algorithm\X;\6
\4${}\}{}$\2\par
\fi

\M{2}\B\X2:Input the graph\X${}\E{}$\6
\&{if} ${}(\\{argc}\I\T{4}){}$\5
${}\{{}$\1\6
${}\\{fprintf}(\\{stderr},\39\.{"Usage:\ \%s\ foo.gb\ so}\)\.{urce\ target%
\\n"},\39\\{argv}[\T{0}]);{}$\6
${}\\{exit}({-}\T{1});{}$\6
\4${}\}{}$\2\6
${}\|g\K\\{restore\_graph}(\\{argv}[\T{1}]);{}$\6
\&{if} ${}(\R\|g){}$\5
${}\{{}$\1\6
${}\\{fprintf}(\\{stderr},\39\.{"I\ can't\ input\ the\ g}\)\.{raph\ \%s\ (panic%
\ code\ }\)\.{\%ld)!\\n"},\39\\{argv}[\T{1}],\39\\{panic\_code});{}$\6
${}\\{exit}({-}\T{2});{}$\6
\4${}\}{}$\2\6
${}\|n\K\|g\MG\|n;{}$\6
\&{if} ${}(\|n>\\{maxn}){}$\5
${}\{{}$\1\6
${}\\{fprintf}(\\{stderr},\39\.{"Sorry,\ that\ graph\ h}\)\.{as\ \%d\ vertices;%
\ "},\39\|n);{}$\6
${}\\{fprintf}(\\{stderr},\39\.{"I\ can't\ handle\ more}\)\.{\ than\ \%d!\\n"},%
\39\\{maxn});{}$\6
${}\\{exit}({-}\T{3});{}$\6
\4${}\}{}$\2\6
\&{if} ${}(\|g\MG\|m>\T{2}*\\{maxm}){}$\5
${}\{{}$\1\6
${}\\{fprintf}(\\{stderr},\39\.{"Sorry,\ that\ graph\ h}\)\.{as\ \%ld\ edges;\
"},\39(\|g\MG\|m+\T{1})/\T{2});{}$\6
${}\\{fprintf}(\\{stderr},\39\.{"I\ can't\ handle\ more}\)\.{\ than\ \%d!\\n"},%
\39\\{maxm});{}$\6
${}\\{exit}({-}\T{3});{}$\6
\4${}\}{}$\2\6
\&{for} ${}(\|v\K\|g\MG\\{vertices};{}$ ${}\|v<\|g\MG\\{vertices}+\|n;{}$ ${}%
\|v\PP){}$\5
${}\{{}$\1\6
\&{if} ${}(\\{strcmp}(\\{argv}[\T{2}],\39\|v\MG\\{name})\E\T{0}){}$\1\5
${}\\{source}\K\|v;{}$\2\6
\&{if} ${}(\\{strcmp}(\\{argv}[\T{3}],\39\|v\MG\\{name})\E\T{0}){}$\1\5
${}\\{target}\K\|v;{}$\2\6
\&{for} ${}(\|a\K\|v\MG\\{arcs};{}$ \|a; ${}\|a\K\|a\MG\\{next}){}$\5
${}\{{}$\1\6
${}\|u\K\|a\MG\\{tip};{}$\6
\&{if} ${}(\|u\E\|v){}$\5
${}\{{}$\1\6
${}\\{fprintf}(\\{stderr},\39\.{"Sorry,\ the\ graph\ co}\)\.{ntains\ a\ loop\ %
\%s--\%s}\)\.{!\\n"},\39\|v\MG\\{name},\39\|v\MG\\{name});{}$\6
${}\\{exit}({-}\T{4});{}$\6
\4${}\}{}$\2\6
${}\|b\K(\|v<\|u\?\|a+\T{1}:\|a-\T{1});{}$\6
\&{if} ${}(\|b\MG\\{tip}\I\|v){}$\5
${}\{{}$\1\6
${}\\{fprintf}(\\{stderr},\39\.{"Sorry,\ the\ graph\ is}\)\.{n't\ undirected!%
\\n"});{}$\6
${}\\{fprintf}(\\{stderr},\39\.{"(\%s->\%s\ has\ mate\ po}\)\.{inting\ to\ \%s)%
\\n"},\39\|v\MG\\{name},\39\|u\MG\\{name},\39\|b\MG\\{tip}\MG\\{name});{}$\6
${}\\{exit}({-}\T{5});{}$\6
\4${}\}{}$\2\6
\4${}\}{}$\2\6
\4${}\}{}$\2\6
\&{if} ${}(\R\\{source}){}$\5
${}\{{}$\1\6
${}\\{fprintf}(\\{stderr},\39\.{"I\ can't\ find\ source}\)\.{\ vertex\ \%s\ in\
the\ gr}\)\.{aph!\\n"},\39\\{argv}[\T{2}]);{}$\6
${}\\{exit}({-}\T{6});{}$\6
\4${}\}{}$\2\6
\&{if} ${}(\R\\{target}){}$\5
${}\{{}$\1\6
${}\\{fprintf}(\\{stderr},\39\.{"I\ can't\ find\ target}\)\.{\ vertex\ \%s\ in\
the\ gr}\)\.{aph!\\n"},\39\\{argv}[\T{3}]);{}$\6
${}\\{exit}({-}\T{7});{}$\6
\4${}\}{}$\2\par
\U1\*.\fi

\M{3}If the source vertex is the first vertex in the graph, I'll process
vertices according to the graph's own ordering.

Otherwise, I use a simple breadth-first strategy to number the vertices:
The source is vertex 1. Then, for each $j\ge1$, I run through the
arcs from vertex~$j$ and assign the first unused number to any of
its neighbors that haven't already got one.

\Y\B\4\D$\\{num}$ \5
$\|z.{}$\|I\par
\Y\B\4\X3:Renumber the vertices\X${}\E{}$\6
\&{if} ${}(\\{source}\E\|g\MG\\{vertices}){}$\5
${}\{{}$\1\6
\&{for} ${}(\|k\K\T{0};{}$ ${}\|k<\|n;{}$ ${}\|k\PP){}$\1\5
${}(\|g\MG\\{vertices}+\|k)\MG\\{num}\K\|k+\T{1},\39\\{vert}[\|k+\T{1}]\K\|g\MG%
\\{vertices}+\|k;{}$\2\6
\4${}\}{}$\5
\2\&{else}\5
${}\{{}$\1\6
\&{for} ${}(\|k\K\T{0};{}$ ${}\|k<\|n;{}$ ${}\|k\PP){}$\1\5
${}(\|g\MG\\{vertices}+\|k)\MG\\{num}\K\T{0};{}$\2\6
${}\\{vert}[\T{1}]\K\\{source},\39\\{source}\MG\\{num}\K\T{1};{}$\6
\&{for} ${}(\|j\K\T{0},\39\|k\K\T{1};{}$ ${}\|j<\|k;{}$ ${}\|j\PP){}$\5
${}\{{}$\1\6
${}\|v\K\\{vert}[\|j+\T{1}];{}$\6
\&{for} ${}(\|a\K\|v\MG\\{arcs};{}$ \|a; ${}\|a\K\|a\MG\\{next}){}$\5
${}\{{}$\1\6
${}\|u\K\|a\MG\\{tip};{}$\6
\&{if} ${}(\|u\MG\\{num}\E\T{0}){}$\1\5
${}\|u\MG\\{num}\K\PP\|k,\39\\{vert}[\|k]\K\|u;{}$\2\6
\4${}\}{}$\2\6
\4${}\}{}$\2\6
\&{if} ${}(\\{target}\MG\\{num}\E\T{0}){}$\5
${}\{{}$\1\6
${}\\{fprintf}(\\{stderr},\39\.{"Sorry,\ there's\ no\ p}\)\.{ath\ from\ \%s\ to%
\ \%s\ in}\)\.{\ the\ graph!\\n"},\39\\{argv}[\T{2}],\39\\{argv}[\T{3}]);{}$\6
${}\\{exit}({-}\T{8});{}$\6
\4${}\}{}$\2\6
\&{if} ${}(\|k<\|n){}$\5
${}\{{}$\1\6
${}\\{fprintf}(\\{stderr},\39\.{"The\ graph\ isn't\ con}\)\.{nected\ (\%d<\%d)!%
\\n"},\39\|k,\39\|n);{}$\6
${}\\{fprintf}(\\{stderr},\39\.{"But\ that's\ OK;\ I'll}\)\.{\ work\ with\ the\
compo}\)\.{nent\ of\ \%s.\\n"},\39\\{argv}[\T{2}]);{}$\6
${}\|n\K\|k;{}$\6
\4${}\}{}$\2\6
\4${}\}{}$\2\par
\U1\*.\fi

\M{4\*}The edges will be considered as arcs $j\to k$ between vertex number~$j$
and vertex number~$k$, when $j<k$ and those vertices are adjacent in
the graph. We process them in order of increasing~$j$; but for fixed~$j$,
the values of~$k$ aren't necessarily increasing.

The $k$ values appear in the \PB{\\{arcto}} array. The edges for fixed~$j$
occur in positions \PB{\\{firstarc}[\|j]} through \PB{$\\{firstarc}[\|j+\T{1}]-%
\T{1}$} of that array.

After this step, we forget the GraphBase data structures and just work
with our homegrown integer-only representation.

\Y\B\4\X4\*:Reformat the edges\X${}\E{}$\6
\&{for} ${}(\|m\K\T{0},\39\|k\K\T{1};{}$ ${}\|k\Z\|n;{}$ ${}\|k\PP){}$\5
${}\{{}$\1\6
${}\\{firstarc}[\|k]\K\|m;{}$\6
${}\|v\K\\{vert}[\|k];{}$\6
${}\\{printf}(\.{"\%ld(\%s)\\n"},\39\|v\MG\\{num},\39\|v\MG\\{name});{}$\6
\&{for} ${}(\|a\K\|v\MG\\{arcs};{}$ \|a; ${}\|a\K\|a\MG\\{next}){}$\5
${}\{{}$\1\6
${}\|u\K\|a\MG\\{tip};{}$\6
\&{if} ${}(\|u\MG\\{num}>\|k){}$\5
${}\{{}$\1\6
${}\\{arcto}[\|m\PP]\K\|u\MG\\{num};{}$\6
\&{if} ${}(\|a\MG\\{len}\E\T{1}){}$\1\5
${}\\{printf}(\.{"\ ->\ \%ld(\%s)\ \#\%d\\n"},\39\|u\MG\\{num},\39\|u\MG%
\\{name},\39\|m);{}$\2\6
\&{else}\1\5
${}\\{printf}(\.{"\ ->\ \%ld(\%s,\%ld)\ \#\%d}\)\.{\\n"},\39\|u\MG\\{num},\39%
\|u\MG\\{name},\39\|a\MG\\{len},\39\|m);{}$\2\6
\4${}\}{}$\2\6
\4${}\}{}$\2\6
${}\\{arcto}[\|m\PP]\K\\{dummy};{}$\6
${}\\{printf}(\.{"\ ->\ dummy,0\ \#\%d\\n"},\39\|m);{}$\6
\4${}\}{}$\2\6
${}\\{firstarc}[\|k]\K\|m{}$;\par
\U1\*.\fi

\N{1}{5}The algorithm.
Now comes the fun part. We systematically construct a binary decision
diagram for all simple paths by working top-down, considering the
arcs in \PB{\\{arcto}}, one by one.

When we're dealing with arc \PB{\|i}, we've already constructed a table of
all possible states that might arise when each of the previous arcs has
been chosen-or-not, except for states that obviously cannot be
part of a simple path.

Arc \PB{\|i} runs from vertex \PB{\|j} to vertex \PB{$\|k\K\\{arcto}[\|i]$}.
Let \PB{\|l} be the maximum vertex number in arcs less than~\PB{\|i}.
(If the breadth-first ordering option was taken above, we'll always
have \PB{$\|k\Z\|l+\T{1}$}, because of the way we did the numbering and
reformatting;
but that method is not always best.)

The state before we decide whether or not to include arc~\PB{\|i} is
represented by a table of values \PB{\\{mate}[\|t]}, for $j\le t\le l$,
with the following significance:
If \PB{$\\{mate}[\|t]\K\|t$}, the previous arcs haven't touched vertex \PB{%
\|t}.
If \PB{$\\{mate}[\|t]\K\|u$} and \PB{$\|u\I\|t$}, the previous arcs have
connected \PB{\|t} with \PB{\|u} by
a simple path.
If \PB{$\\{mate}[\|t]\K\T{0}$}, the previous arcs have ``saturated'' vertex~%
\PB{\|t}; we can't
touch it again.

We also use a (slick?)\ trick: We imagine that an edge between the
source and target has already been included. Then the final arc of
a simple path will be an arc that completes a cycle, when no other
incomplete paths are present. (Think about it.)

The \PB{\\{mate}} information is all that we need to know about the behavior of
previous arcs. And it's easily updated when we add the \PB{\|i}th arc (or not).
So each ``state'' is equivalent to a \PB{\\{mate}} table, consisting of
\PB{$\|l+\T{1}-\|j$} numbers.

The states are stored in a queue, indexed by 64-bit numbers
\PB{\\{tail}}, \PB{\\{boundary}}, and \PB{\\{head}}, where \PB{$\\{tail}\Z%
\\{boundary}\Z\\{head}$}.
Between \PB{\\{tail}} and \PB{\\{boundary}} are the pre-arc-\PB{\|i} states
that haven't yet
been processed; between \PB{\\{boundary}} and \PB{\\{head}} are the post-arc-%
\PB{\|i} states
that will be considered later. The states before \PB{\\{boundary}}
are sequences of \PB{$\|s\K\|l+\T{1}-\|j$} bytes each, and the states after %
\PB{\\{boundary}}
are sequences of \PB{$\\{ss}\K\\{ll}+\T{1}-\\{jj}$} bytes each, where \PB{%
\\{ll}} and \PB{\\{jj}} are the values of
\PB{\|l} and \PB{\|j} for arc \PB{$\|i+\T{1}$}.

Bytes of the queue are stored in \PB{\\{mem}}, which wraps around modulo \PB{%
\\{memsize}}.
We ensure that \PB{$\\{head}-\\{tail}$} never exceeds \PB{\\{memsize}}.


\Y\B\4\X5:Do the algorithm\X${}\E{}$\6
\X6\*:Initialize the \PB{\\{mate}} table\X;\6
\X7:Initialize the queue\X;\6
\&{for} ${}(\|i\K\T{0};{}$ ${}\|i<\|m;{}$ ${}\|i\PP){}$\5
${}\{{}$\1\6
${}\\{printf}(\.{"\#\%d:\\n"},\39\|i+\T{1}){}$;\C{ announce that we're
beginning a new arc }\6
${}\\{fprintf}(\\{stderr},\39\.{"Beginning\ arc\ \%d\ (s}\)\.{erial=%
\%d,head-tail=\%}\)\.{lld)\\n"},\39\|i+\T{1},\39\\{serial},\39\\{head}-%
\\{tail});{}$\6
\\{fflush}(\\{stderr});\6
\X8\*:Process arc \PB{\|i}\X;\6
\4${}\}{}$\2\par
\U1\*.\fi

\M{6\*}\B\X6\*:Initialize the \PB{\\{mate}} table\X${}\E{}$\6
\&{for} ${}(\|t\K\T{2};{}$ ${}\|t\Z\|n;{}$ ${}\|t\PP){}$\1\5
${}\\{mate}[\|t]\K\|t;{}$\2\6
${}\\{mate}[\\{dummy}]\K\T{1},\39\\{mate}[\T{1}]\K\\{dummy}{}$;\par
\U5.\fi

\M{7}\B\X7:Initialize the queue\X${}\E{}$\6
$\\{jj}\K\\{ll}\K\T{1};{}$\6
${}\\{mem}[\T{0}]\K\\{mate}[\T{1}];{}$\6
${}\\{tail}\K\T{0},\39\\{head}\K\T{1};{}$\6
${}\\{serial}\K\T{2}{}$;\par
\U5.\fi

\M{8\*}Each state for a particular arc gets a distinguishing number.
Two states are special: 0 means the losing state, when a simple path
is impossible; 1 means the winning state, when a simple path has been
completed. The other states are 2 or more.

The output format on \PB{\\{stdout}} simply shows the identifying numbers of a
state
and its two successors, in hexadecimal.

\Y\B\4\D$\\{trunc}(\\{addr})$ \5
$((\\{addr})\AND(\\{memsize}-\T{1}){}$)\par
\Y\B\4\X8\*:Process arc \PB{\|i}\X${}\E{}$\6
$\\{boundary}\K\\{head},\39\\{htcount}\K\T{0},\39\\{htid}\K(\|i+\\{wrap})\LL%
\\{logmemsize};{}$\6
\&{if} ${}(\\{htid}\E\T{0}){}$\5
${}\{{}$\1\6
\&{for} ${}(\\{hash}\K\T{0};{}$ ${}\\{hash}<\\{htsize};{}$ ${}\\{hash}\PP){}$\1%
\5
${}\\{htable}[\\{hash}]\K\T{0};{}$\2\6
${}\\{wrap}\PP,\39\\{htid}\K\T{1}\LL\\{logmemsize};{}$\6
\4${}\}{}$\2\6
${}\\{newserial}\K\\{serial}+((\\{head}-\\{tail})/(\\{ll}+\T{1}-\\{jj}));{}$\6
${}\|j\K\\{jj},\39\|k\K\\{arcto}[\|i],\39\|l\K\\{ll};{}$\6
\&{while} ${}(\\{jj}\Z\|n\W\\{firstarc}[\\{jj}+\T{1}]\E\|i+\T{1}){}$\1\5
${}\\{jj}\PP;{}$\2\6
${}\\{ll}\K(\|k\E\\{dummy}\?\|l:\|k>\|l\?\|k:\|l);{}$\6
\&{while} ${}(\\{tail}<\\{boundary}){}$\5
${}\{{}$\1\6
${}\\{printf}(\.{"\%x:"},\39\\{serial});{}$\6
${}\\{serial}\PP;{}$\6
\X9\*:Unpack a state, and move \PB{\\{tail}} up\X;\6
\X11:Print the successor if arc \PB{\|i} is not chosen\X;\6
\\{printf}(\.{","});\6
\X10:Print the successor if arc \PB{\|i} is chosen\X;\6
\\{printf}(\.{"\\n"});\6
\4${}\}{}$\2\par
\U5.\fi

\M{9\*}If the target vertex hasn't entered the action yet (that is, if it
exceeds~\PB{\|l}), we must update its \PB{\\{mate}} entry at this point.

\Y\B\4\X9\*:Unpack a state, and move \PB{\\{tail}} up\X${}\E{}$\6
$\\{mate}[\\{dummy}]\K\T{0};{}$\6
\&{for} ${}(\|t\K\|j;{}$ ${}\|t\Z\|l;{}$ ${}\|t\PP,\39\\{tail}\PP){}$\5
${}\{{}$\1\6
${}\\{mate}[\|t]\K\\{mem}[\\{trunc}(\\{tail})];{}$\6
\&{if} ${}(\\{mate}[\|t]\E\\{dummy}){}$\1\5
${}\\{mate}[\\{dummy}]\K\|t;{}$\2\6
\4${}\}{}$\2\par
\U8\*.\fi

\M{10}Here's where we update the mates. The order of doing this is carefully
chosen so that it works fine when \PB{$\\{mate}[\|j]\K\|j$} and/or \PB{$%
\\{mate}[\|k]\K\|k$}.

\Y\B\4\X10:Print the successor if arc \PB{\|i} is chosen\X${}\E{}$\6
$\\{jm}\K\\{mate}[\|j],\39\\{km}\K\\{mate}[\|k];{}$\6
\&{if} ${}(\\{jm}\E\T{0}\V\\{km}\E\T{0}){}$\1\5
\\{printf}(\.{"0"});\C{ we mustn't touch a saturated vertex }\2\6
\&{else} \&{if} ${}(\\{jm}\E\|k){}$\1\5
\X12\*:Print 1 or 0, depending on whether this arc wins or loses\X\2\6
\&{else}\5
${}\{{}$\1\6
${}\\{mate}[\|j]\K\T{0},\39\\{mate}[\|k]\K\T{0};{}$\6
${}\\{mate}[\\{jm}]\K\\{km},\39\\{mate}[\\{km}]\K\\{jm};{}$\6
${}\\{printstate}(\|j,\39\\{jj},\39\\{ll});{}$\6
${}\\{mate}[\\{jm}]\K\|j,\39\\{mate}[\\{km}]\K\|k,\39\\{mate}[\|j]\K\\{jm},\39%
\\{mate}[\|k]\K\\{km}{}$;\C{ restore original state }\6
\4${}\}{}$\2\6
\\{done}:\par
\U8\*.\fi

\M{11}\B\X11:Print the successor if arc \PB{\|i} is not chosen\X${}\E{}$\6
$\\{printstate}(\|j,\39\\{jj},\39\\{ll}){}$;\par
\U8\*.\fi

\M{12\*}See the note below regarding a change that will restrict consideration
to Hamiltonian paths. A similar change is needed here.

\Y\B\4\X12\*:Print 1 or 0, depending on whether this arc wins or loses\X${}%
\E{}$\6
${}\{{}$\1\6
\&{for} ${}(\|t\K\|j+\T{1};{}$ ${}\|t\Z\\{ll};{}$ ${}\|t\PP){}$\1\6
\&{if} ${}(\|t\I\|k){}$\5
${}\{{}$\1\6
\&{if} (\\{mate}[\|t])\1\5
\&{break};\2\6
\4${}\}{}$\2\2\6
\&{if} ${}(\|t>\|n){}$\1\5
\\{printf}(\.{"1"});\C{ we win: this cycle is all by itself }\2\6
\&{else}\1\5
\\{printf}(\.{"0"});\C{ we lose: there's junk outside this cycle }\2\6
\4${}\}{}$\2\par
\U10.\fi

\M{13\*}The \PB{\\{printstate}} subroutine does the rest of the work. It makes
sure
that no incomplete paths linger in positions \PB{\|j} through \PB{$\\{jj}-%
\T{1}$}, which
are about to disappear; and it puts the contents of \PB{\\{mate}[\\{jj}]}
through
\PB{\\{mate}[\\{ll}]} into the queue, checking to see if it was already there.

If `\PB{$\\{mate}[\|t]\I\|t$}' is removed from the condition below, we get
Hamiltonian paths only (I mean, simple paths that include every vertex).

\Y\B\4\X13\*:Subroutines\X${}\E{}$\6
\1\1\&{void} \\{printstate}(\&{int} \|j${},\39{}$\&{int} \\{jj}${},\39{}$%
\&{int} \\{ll})\2\2\6
${}\{{}$\1\6
\&{register} \&{int} \|h${},\39\\{hh},\39\\{ss},\39\|t,\39\\{tt},\39%
\\{hash};{}$\7
\&{for} ${}(\|t\K\|j;{}$ ${}\|t<\\{jj};{}$ ${}\|t\PP){}$\1\6
\&{if} (\\{mate}[\|t])\1\5
\&{break};\2\2\6
\&{if} ${}(\|t<\\{jj}){}$\1\5
\\{printf}(\.{"0"});\C{ incomplete junk mustn't be left hanging }\2\6
\&{else} \&{if} ${}(\\{ll}<\\{jj}){}$\1\5
\\{printf}(\.{"0"});\C{ nothing is viable }\2\6
\&{else}\5
${}\{{}$\1\6
${}\\{ss}\K\\{ll}+\T{1}-\\{jj};{}$\6
\&{if} ${}(\\{head}+\\{ss}-\\{tail}>\\{memsize}){}$\5
${}\{{}$\1\6
${}\\{fprintf}(\\{stderr},\39\.{"Oops,\ I'm\ out\ of\ me}\)\.{mory\ (memsize=%
\%d,\ se}\)\.{rial=\%d)!\\n"},\39\\{memsize},\39\\{serial});{}$\6
\\{fflush}(\\{stdout});\6
${}\\{exit}({-}\T{69});{}$\6
\4${}\}{}$\2\6
\X14:Move the current state into position after \PB{\\{head}}, and compute \PB{%
\\{hash}}\X;\6
\X15:Find the first match, \PB{\\{hh}}, for the current state after \PB{%
\\{boundary}}\X;\6
${}\|h\K\\{trunc}(\\{hh}-\\{boundary})/\\{ss};{}$\6
${}\\{printf}(\.{"\%x"},\39\\{newserial}+\|h);{}$\6
\4${}\}{}$\2\6
\4${}\}{}$\2\par
\U1\*.\fi

\M{14}\B\X14:Move the current state into position after \PB{\\{head}}, and
compute \PB{\\{hash}}\X${}\E{}$\6
\&{for} ${}(\|t\K\\{jj},\39\|h\K\\{trunc}(\\{head}),\39\\{hash}\K\T{0};{}$ ${}%
\|t\Z\\{ll};{}$ ${}\|t\PP,\39\|h\K\\{trunc}(\|h+\T{1})){}$\5
${}\{{}$\1\6
${}\\{mem}[\|h]\K\\{mate}[\|t];{}$\6
${}\\{hash}\K\\{hash}*\T{31415926525}+\\{mate}[\|t];{}$\6
\4${}\}{}$\2\par
\U13\*.\fi

\M{15}The hash table is automatically cleared whenever \PB{\\{htid}} is
increased,
because we store \PB{\\{htid}} with each relevant table entry.

\Y\B\4\X15:Find the first match, \PB{\\{hh}}, for the current state after \PB{%
\\{boundary}}\X${}\E{}$\6
\&{for} ${}(\\{hash}\K\\{hash}\AND(\\{htsize}-\T{1});{}$  ; ${}\\{hash}\K(%
\\{hash}+\T{1})\AND(\\{htsize}-\T{1})){}$\5
${}\{{}$\1\6
${}\\{hh}\K\\{htable}[\\{hash}];{}$\6
\&{if} ${}((\\{hh}\XOR\\{htid})\G\\{memsize}){}$\1\5
\X16:Insert new entry and \PB{\&{goto} \\{found}}\X;\2\6
${}\\{hh}\K\\{trunc}(\\{hh});{}$\6
\&{for} ${}(\|t\K\\{hh},\39\|h\K\\{trunc}(\\{head}),\39\\{tt}\K\\{trunc}(\|t+%
\\{ss}-\T{1});{}$  ; ${}\|t\K\\{trunc}(\|t+\T{1}),\39\|h\K\\{trunc}(\|h+%
\T{1})){}$\5
${}\{{}$\1\6
\&{if} ${}(\\{mem}[\|t]\I\\{mem}[\|h]){}$\1\5
\&{break};\2\6
\&{if} ${}(\|t\E\\{tt}){}$\1\5
\&{goto} \\{found};\2\6
\4${}\}{}$\2\6
\4${}\}{}$\2\6
\\{found}:\par
\U13\*.\fi

\M{16}\B\X16:Insert new entry and \PB{\&{goto} \\{found}}\X${}\E{}$\6
${}\{{}$\1\6
\&{if} ${}(\PP\\{htcount}>(\\{htsize}\GG\T{1})){}$\5
${}\{{}$\1\6
${}\\{fprintf}(\\{stderr},\39\.{"Sorry,\ the\ hash\ tab}\)\.{le\ is\ full\
(htsize=\%}\)\.{d,\ serial=\%d)!\\n"},\39\\{htsize},\39\\{serial});{}$\6
${}\\{exit}({-}\T{96});{}$\6
\4${}\}{}$\2\6
${}\\{hh}\K\\{trunc}(\\{head});{}$\6
${}\\{htable}[\\{hash}]\K\\{htid}+\\{hh};{}$\6
${}\\{head}\MRL{+{\K}}\\{ss};{}$\6
\&{goto} \\{found};\6
\4${}\}{}$\2\par
\U15.\fi

\N{1}{17\*}Index.
\fi

\ch 1\*, 4\*, 6\*, 8\*, 9\*, 12\*, 13\*, 17\*.

\inx
\fin
\con
